
% Default to the notebook output style

    


% Inherit from the specified cell style.




    
\documentclass[11pt]{article}

    
    
    \usepackage[T1]{fontenc}
    % Nicer default font (+ math font) than Computer Modern for most use cases
    \usepackage{mathpazo}

    % Basic figure setup, for now with no caption control since it's done
    % automatically by Pandoc (which extracts ![](path) syntax from Markdown).
    \usepackage{graphicx}
    % We will generate all images so they have a width \maxwidth. This means
    % that they will get their normal width if they fit onto the page, but
    % are scaled down if they would overflow the margins.
    \makeatletter
    \def\maxwidth{\ifdim\Gin@nat@width>\linewidth\linewidth
    \else\Gin@nat@width\fi}
    \makeatother
    \let\Oldincludegraphics\includegraphics
    % Set max figure width to be 80% of text width, for now hardcoded.
    \renewcommand{\includegraphics}[1]{\Oldincludegraphics[width=.8\maxwidth]{#1}}
    % Ensure that by default, figures have no caption (until we provide a
    % proper Figure object with a Caption API and a way to capture that
    % in the conversion process - todo).
    \usepackage{caption}
    \DeclareCaptionLabelFormat{nolabel}{}
    \captionsetup{labelformat=nolabel}

    \usepackage{adjustbox} % Used to constrain images to a maximum size 
    \usepackage{xcolor} % Allow colors to be defined
    \usepackage{enumerate} % Needed for markdown enumerations to work
    \usepackage{geometry} % Used to adjust the document margins
    \usepackage{amsmath} % Equations
    \usepackage{amssymb} % Equations
    \usepackage{textcomp} % defines textquotesingle
    % Hack from http://tex.stackexchange.com/a/47451/13684:
    \AtBeginDocument{%
        \def\PYZsq{\textquotesingle}% Upright quotes in Pygmentized code
    }
    \usepackage{upquote} % Upright quotes for verbatim code
    \usepackage{eurosym} % defines \euro
    \usepackage[mathletters]{ucs} % Extended unicode (utf-8) support
    \usepackage[utf8x]{inputenc} % Allow utf-8 characters in the tex document
    \usepackage{fancyvrb} % verbatim replacement that allows latex
    \usepackage{grffile} % extends the file name processing of package graphics 
                         % to support a larger range 
    % The hyperref package gives us a pdf with properly built
    % internal navigation ('pdf bookmarks' for the table of contents,
    % internal cross-reference links, web links for URLs, etc.)
    \usepackage{hyperref}
    \usepackage{longtable} % longtable support required by pandoc >1.10
    \usepackage{booktabs}  % table support for pandoc > 1.12.2
    \usepackage[inline]{enumitem} % IRkernel/repr support (it uses the enumerate* environment)
    \usepackage[normalem]{ulem} % ulem is needed to support strikethroughs (\sout)
                                % normalem makes italics be italics, not underlines
    

    
    
    % Colors for the hyperref package
    \definecolor{urlcolor}{rgb}{0,.145,.698}
    \definecolor{linkcolor}{rgb}{.71,0.21,0.01}
    \definecolor{citecolor}{rgb}{.12,.54,.11}

    % ANSI colors
    \definecolor{ansi-black}{HTML}{3E424D}
    \definecolor{ansi-black-intense}{HTML}{282C36}
    \definecolor{ansi-red}{HTML}{E75C58}
    \definecolor{ansi-red-intense}{HTML}{B22B31}
    \definecolor{ansi-green}{HTML}{00A250}
    \definecolor{ansi-green-intense}{HTML}{007427}
    \definecolor{ansi-yellow}{HTML}{DDB62B}
    \definecolor{ansi-yellow-intense}{HTML}{B27D12}
    \definecolor{ansi-blue}{HTML}{208FFB}
    \definecolor{ansi-blue-intense}{HTML}{0065CA}
    \definecolor{ansi-magenta}{HTML}{D160C4}
    \definecolor{ansi-magenta-intense}{HTML}{A03196}
    \definecolor{ansi-cyan}{HTML}{60C6C8}
    \definecolor{ansi-cyan-intense}{HTML}{258F8F}
    \definecolor{ansi-white}{HTML}{C5C1B4}
    \definecolor{ansi-white-intense}{HTML}{A1A6B2}

    % commands and environments needed by pandoc snippets
    % extracted from the output of `pandoc -s`
    \providecommand{\tightlist}{%
      \setlength{\itemsep}{0pt}\setlength{\parskip}{0pt}}
    \DefineVerbatimEnvironment{Highlighting}{Verbatim}{commandchars=\\\{\}}
    % Add ',fontsize=\small' for more characters per line
    \newenvironment{Shaded}{}{}
    \newcommand{\KeywordTok}[1]{\textcolor[rgb]{0.00,0.44,0.13}{\textbf{{#1}}}}
    \newcommand{\DataTypeTok}[1]{\textcolor[rgb]{0.56,0.13,0.00}{{#1}}}
    \newcommand{\DecValTok}[1]{\textcolor[rgb]{0.25,0.63,0.44}{{#1}}}
    \newcommand{\BaseNTok}[1]{\textcolor[rgb]{0.25,0.63,0.44}{{#1}}}
    \newcommand{\FloatTok}[1]{\textcolor[rgb]{0.25,0.63,0.44}{{#1}}}
    \newcommand{\CharTok}[1]{\textcolor[rgb]{0.25,0.44,0.63}{{#1}}}
    \newcommand{\StringTok}[1]{\textcolor[rgb]{0.25,0.44,0.63}{{#1}}}
    \newcommand{\CommentTok}[1]{\textcolor[rgb]{0.38,0.63,0.69}{\textit{{#1}}}}
    \newcommand{\OtherTok}[1]{\textcolor[rgb]{0.00,0.44,0.13}{{#1}}}
    \newcommand{\AlertTok}[1]{\textcolor[rgb]{1.00,0.00,0.00}{\textbf{{#1}}}}
    \newcommand{\FunctionTok}[1]{\textcolor[rgb]{0.02,0.16,0.49}{{#1}}}
    \newcommand{\RegionMarkerTok}[1]{{#1}}
    \newcommand{\ErrorTok}[1]{\textcolor[rgb]{1.00,0.00,0.00}{\textbf{{#1}}}}
    \newcommand{\NormalTok}[1]{{#1}}
    
    % Additional commands for more recent versions of Pandoc
    \newcommand{\ConstantTok}[1]{\textcolor[rgb]{0.53,0.00,0.00}{{#1}}}
    \newcommand{\SpecialCharTok}[1]{\textcolor[rgb]{0.25,0.44,0.63}{{#1}}}
    \newcommand{\VerbatimStringTok}[1]{\textcolor[rgb]{0.25,0.44,0.63}{{#1}}}
    \newcommand{\SpecialStringTok}[1]{\textcolor[rgb]{0.73,0.40,0.53}{{#1}}}
    \newcommand{\ImportTok}[1]{{#1}}
    \newcommand{\DocumentationTok}[1]{\textcolor[rgb]{0.73,0.13,0.13}{\textit{{#1}}}}
    \newcommand{\AnnotationTok}[1]{\textcolor[rgb]{0.38,0.63,0.69}{\textbf{\textit{{#1}}}}}
    \newcommand{\CommentVarTok}[1]{\textcolor[rgb]{0.38,0.63,0.69}{\textbf{\textit{{#1}}}}}
    \newcommand{\VariableTok}[1]{\textcolor[rgb]{0.10,0.09,0.49}{{#1}}}
    \newcommand{\ControlFlowTok}[1]{\textcolor[rgb]{0.00,0.44,0.13}{\textbf{{#1}}}}
    \newcommand{\OperatorTok}[1]{\textcolor[rgb]{0.40,0.40,0.40}{{#1}}}
    \newcommand{\BuiltInTok}[1]{{#1}}
    \newcommand{\ExtensionTok}[1]{{#1}}
    \newcommand{\PreprocessorTok}[1]{\textcolor[rgb]{0.74,0.48,0.00}{{#1}}}
    \newcommand{\AttributeTok}[1]{\textcolor[rgb]{0.49,0.56,0.16}{{#1}}}
    \newcommand{\InformationTok}[1]{\textcolor[rgb]{0.38,0.63,0.69}{\textbf{\textit{{#1}}}}}
    \newcommand{\WarningTok}[1]{\textcolor[rgb]{0.38,0.63,0.69}{\textbf{\textit{{#1}}}}}
    
    
    % Define a nice break command that doesn't care if a line doesn't already
    % exist.
    \def\br{\hspace*{\fill} \\* }
    % Math Jax compatability definitions
    \def\gt{>}
    \def\lt{<}
    % Document parameters
    \title{EDA\_and\_correlation\_model}
    
    
    

    % Pygments definitions
    
\makeatletter
\def\PY@reset{\let\PY@it=\relax \let\PY@bf=\relax%
    \let\PY@ul=\relax \let\PY@tc=\relax%
    \let\PY@bc=\relax \let\PY@ff=\relax}
\def\PY@tok#1{\csname PY@tok@#1\endcsname}
\def\PY@toks#1+{\ifx\relax#1\empty\else%
    \PY@tok{#1}\expandafter\PY@toks\fi}
\def\PY@do#1{\PY@bc{\PY@tc{\PY@ul{%
    \PY@it{\PY@bf{\PY@ff{#1}}}}}}}
\def\PY#1#2{\PY@reset\PY@toks#1+\relax+\PY@do{#2}}

\expandafter\def\csname PY@tok@w\endcsname{\def\PY@tc##1{\textcolor[rgb]{0.73,0.73,0.73}{##1}}}
\expandafter\def\csname PY@tok@c\endcsname{\let\PY@it=\textit\def\PY@tc##1{\textcolor[rgb]{0.25,0.50,0.50}{##1}}}
\expandafter\def\csname PY@tok@cp\endcsname{\def\PY@tc##1{\textcolor[rgb]{0.74,0.48,0.00}{##1}}}
\expandafter\def\csname PY@tok@k\endcsname{\let\PY@bf=\textbf\def\PY@tc##1{\textcolor[rgb]{0.00,0.50,0.00}{##1}}}
\expandafter\def\csname PY@tok@kp\endcsname{\def\PY@tc##1{\textcolor[rgb]{0.00,0.50,0.00}{##1}}}
\expandafter\def\csname PY@tok@kt\endcsname{\def\PY@tc##1{\textcolor[rgb]{0.69,0.00,0.25}{##1}}}
\expandafter\def\csname PY@tok@o\endcsname{\def\PY@tc##1{\textcolor[rgb]{0.40,0.40,0.40}{##1}}}
\expandafter\def\csname PY@tok@ow\endcsname{\let\PY@bf=\textbf\def\PY@tc##1{\textcolor[rgb]{0.67,0.13,1.00}{##1}}}
\expandafter\def\csname PY@tok@nb\endcsname{\def\PY@tc##1{\textcolor[rgb]{0.00,0.50,0.00}{##1}}}
\expandafter\def\csname PY@tok@nf\endcsname{\def\PY@tc##1{\textcolor[rgb]{0.00,0.00,1.00}{##1}}}
\expandafter\def\csname PY@tok@nc\endcsname{\let\PY@bf=\textbf\def\PY@tc##1{\textcolor[rgb]{0.00,0.00,1.00}{##1}}}
\expandafter\def\csname PY@tok@nn\endcsname{\let\PY@bf=\textbf\def\PY@tc##1{\textcolor[rgb]{0.00,0.00,1.00}{##1}}}
\expandafter\def\csname PY@tok@ne\endcsname{\let\PY@bf=\textbf\def\PY@tc##1{\textcolor[rgb]{0.82,0.25,0.23}{##1}}}
\expandafter\def\csname PY@tok@nv\endcsname{\def\PY@tc##1{\textcolor[rgb]{0.10,0.09,0.49}{##1}}}
\expandafter\def\csname PY@tok@no\endcsname{\def\PY@tc##1{\textcolor[rgb]{0.53,0.00,0.00}{##1}}}
\expandafter\def\csname PY@tok@nl\endcsname{\def\PY@tc##1{\textcolor[rgb]{0.63,0.63,0.00}{##1}}}
\expandafter\def\csname PY@tok@ni\endcsname{\let\PY@bf=\textbf\def\PY@tc##1{\textcolor[rgb]{0.60,0.60,0.60}{##1}}}
\expandafter\def\csname PY@tok@na\endcsname{\def\PY@tc##1{\textcolor[rgb]{0.49,0.56,0.16}{##1}}}
\expandafter\def\csname PY@tok@nt\endcsname{\let\PY@bf=\textbf\def\PY@tc##1{\textcolor[rgb]{0.00,0.50,0.00}{##1}}}
\expandafter\def\csname PY@tok@nd\endcsname{\def\PY@tc##1{\textcolor[rgb]{0.67,0.13,1.00}{##1}}}
\expandafter\def\csname PY@tok@s\endcsname{\def\PY@tc##1{\textcolor[rgb]{0.73,0.13,0.13}{##1}}}
\expandafter\def\csname PY@tok@sd\endcsname{\let\PY@it=\textit\def\PY@tc##1{\textcolor[rgb]{0.73,0.13,0.13}{##1}}}
\expandafter\def\csname PY@tok@si\endcsname{\let\PY@bf=\textbf\def\PY@tc##1{\textcolor[rgb]{0.73,0.40,0.53}{##1}}}
\expandafter\def\csname PY@tok@se\endcsname{\let\PY@bf=\textbf\def\PY@tc##1{\textcolor[rgb]{0.73,0.40,0.13}{##1}}}
\expandafter\def\csname PY@tok@sr\endcsname{\def\PY@tc##1{\textcolor[rgb]{0.73,0.40,0.53}{##1}}}
\expandafter\def\csname PY@tok@ss\endcsname{\def\PY@tc##1{\textcolor[rgb]{0.10,0.09,0.49}{##1}}}
\expandafter\def\csname PY@tok@sx\endcsname{\def\PY@tc##1{\textcolor[rgb]{0.00,0.50,0.00}{##1}}}
\expandafter\def\csname PY@tok@m\endcsname{\def\PY@tc##1{\textcolor[rgb]{0.40,0.40,0.40}{##1}}}
\expandafter\def\csname PY@tok@gh\endcsname{\let\PY@bf=\textbf\def\PY@tc##1{\textcolor[rgb]{0.00,0.00,0.50}{##1}}}
\expandafter\def\csname PY@tok@gu\endcsname{\let\PY@bf=\textbf\def\PY@tc##1{\textcolor[rgb]{0.50,0.00,0.50}{##1}}}
\expandafter\def\csname PY@tok@gd\endcsname{\def\PY@tc##1{\textcolor[rgb]{0.63,0.00,0.00}{##1}}}
\expandafter\def\csname PY@tok@gi\endcsname{\def\PY@tc##1{\textcolor[rgb]{0.00,0.63,0.00}{##1}}}
\expandafter\def\csname PY@tok@gr\endcsname{\def\PY@tc##1{\textcolor[rgb]{1.00,0.00,0.00}{##1}}}
\expandafter\def\csname PY@tok@ge\endcsname{\let\PY@it=\textit}
\expandafter\def\csname PY@tok@gs\endcsname{\let\PY@bf=\textbf}
\expandafter\def\csname PY@tok@gp\endcsname{\let\PY@bf=\textbf\def\PY@tc##1{\textcolor[rgb]{0.00,0.00,0.50}{##1}}}
\expandafter\def\csname PY@tok@go\endcsname{\def\PY@tc##1{\textcolor[rgb]{0.53,0.53,0.53}{##1}}}
\expandafter\def\csname PY@tok@gt\endcsname{\def\PY@tc##1{\textcolor[rgb]{0.00,0.27,0.87}{##1}}}
\expandafter\def\csname PY@tok@err\endcsname{\def\PY@bc##1{\setlength{\fboxsep}{0pt}\fcolorbox[rgb]{1.00,0.00,0.00}{1,1,1}{\strut ##1}}}
\expandafter\def\csname PY@tok@kc\endcsname{\let\PY@bf=\textbf\def\PY@tc##1{\textcolor[rgb]{0.00,0.50,0.00}{##1}}}
\expandafter\def\csname PY@tok@kd\endcsname{\let\PY@bf=\textbf\def\PY@tc##1{\textcolor[rgb]{0.00,0.50,0.00}{##1}}}
\expandafter\def\csname PY@tok@kn\endcsname{\let\PY@bf=\textbf\def\PY@tc##1{\textcolor[rgb]{0.00,0.50,0.00}{##1}}}
\expandafter\def\csname PY@tok@kr\endcsname{\let\PY@bf=\textbf\def\PY@tc##1{\textcolor[rgb]{0.00,0.50,0.00}{##1}}}
\expandafter\def\csname PY@tok@bp\endcsname{\def\PY@tc##1{\textcolor[rgb]{0.00,0.50,0.00}{##1}}}
\expandafter\def\csname PY@tok@fm\endcsname{\def\PY@tc##1{\textcolor[rgb]{0.00,0.00,1.00}{##1}}}
\expandafter\def\csname PY@tok@vc\endcsname{\def\PY@tc##1{\textcolor[rgb]{0.10,0.09,0.49}{##1}}}
\expandafter\def\csname PY@tok@vg\endcsname{\def\PY@tc##1{\textcolor[rgb]{0.10,0.09,0.49}{##1}}}
\expandafter\def\csname PY@tok@vi\endcsname{\def\PY@tc##1{\textcolor[rgb]{0.10,0.09,0.49}{##1}}}
\expandafter\def\csname PY@tok@vm\endcsname{\def\PY@tc##1{\textcolor[rgb]{0.10,0.09,0.49}{##1}}}
\expandafter\def\csname PY@tok@sa\endcsname{\def\PY@tc##1{\textcolor[rgb]{0.73,0.13,0.13}{##1}}}
\expandafter\def\csname PY@tok@sb\endcsname{\def\PY@tc##1{\textcolor[rgb]{0.73,0.13,0.13}{##1}}}
\expandafter\def\csname PY@tok@sc\endcsname{\def\PY@tc##1{\textcolor[rgb]{0.73,0.13,0.13}{##1}}}
\expandafter\def\csname PY@tok@dl\endcsname{\def\PY@tc##1{\textcolor[rgb]{0.73,0.13,0.13}{##1}}}
\expandafter\def\csname PY@tok@s2\endcsname{\def\PY@tc##1{\textcolor[rgb]{0.73,0.13,0.13}{##1}}}
\expandafter\def\csname PY@tok@sh\endcsname{\def\PY@tc##1{\textcolor[rgb]{0.73,0.13,0.13}{##1}}}
\expandafter\def\csname PY@tok@s1\endcsname{\def\PY@tc##1{\textcolor[rgb]{0.73,0.13,0.13}{##1}}}
\expandafter\def\csname PY@tok@mb\endcsname{\def\PY@tc##1{\textcolor[rgb]{0.40,0.40,0.40}{##1}}}
\expandafter\def\csname PY@tok@mf\endcsname{\def\PY@tc##1{\textcolor[rgb]{0.40,0.40,0.40}{##1}}}
\expandafter\def\csname PY@tok@mh\endcsname{\def\PY@tc##1{\textcolor[rgb]{0.40,0.40,0.40}{##1}}}
\expandafter\def\csname PY@tok@mi\endcsname{\def\PY@tc##1{\textcolor[rgb]{0.40,0.40,0.40}{##1}}}
\expandafter\def\csname PY@tok@il\endcsname{\def\PY@tc##1{\textcolor[rgb]{0.40,0.40,0.40}{##1}}}
\expandafter\def\csname PY@tok@mo\endcsname{\def\PY@tc##1{\textcolor[rgb]{0.40,0.40,0.40}{##1}}}
\expandafter\def\csname PY@tok@ch\endcsname{\let\PY@it=\textit\def\PY@tc##1{\textcolor[rgb]{0.25,0.50,0.50}{##1}}}
\expandafter\def\csname PY@tok@cm\endcsname{\let\PY@it=\textit\def\PY@tc##1{\textcolor[rgb]{0.25,0.50,0.50}{##1}}}
\expandafter\def\csname PY@tok@cpf\endcsname{\let\PY@it=\textit\def\PY@tc##1{\textcolor[rgb]{0.25,0.50,0.50}{##1}}}
\expandafter\def\csname PY@tok@c1\endcsname{\let\PY@it=\textit\def\PY@tc##1{\textcolor[rgb]{0.25,0.50,0.50}{##1}}}
\expandafter\def\csname PY@tok@cs\endcsname{\let\PY@it=\textit\def\PY@tc##1{\textcolor[rgb]{0.25,0.50,0.50}{##1}}}

\def\PYZbs{\char`\\}
\def\PYZus{\char`\_}
\def\PYZob{\char`\{}
\def\PYZcb{\char`\}}
\def\PYZca{\char`\^}
\def\PYZam{\char`\&}
\def\PYZlt{\char`\<}
\def\PYZgt{\char`\>}
\def\PYZsh{\char`\#}
\def\PYZpc{\char`\%}
\def\PYZdl{\char`\$}
\def\PYZhy{\char`\-}
\def\PYZsq{\char`\'}
\def\PYZdq{\char`\"}
\def\PYZti{\char`\~}
% for compatibility with earlier versions
\def\PYZat{@}
\def\PYZlb{[}
\def\PYZrb{]}
\makeatother


    % Exact colors from NB
    \definecolor{incolor}{rgb}{0.0, 0.0, 0.5}
    \definecolor{outcolor}{rgb}{0.545, 0.0, 0.0}



    
    % Prevent overflowing lines due to hard-to-break entities
    \sloppy 
    % Setup hyperref package
    \hypersetup{
      breaklinks=true,  % so long urls are correctly broken across lines
      colorlinks=true,
      urlcolor=urlcolor,
      linkcolor=linkcolor,
      citecolor=citecolor,
      }
    % Slightly bigger margins than the latex defaults
    
    \geometry{verbose,tmargin=1in,bmargin=1in,lmargin=1in,rmargin=1in}
    
    

    \begin{document}
    
    
    \maketitle
    
    

    
    \begin{Verbatim}[commandchars=\\\{\}]
{\color{incolor}In [{\color{incolor}1}]:} \PY{c+c1}{\PYZsh{} Importing standard libraries}
        \PY{k+kn}{import} \PY{n+nn}{pandas} \PY{k}{as} \PY{n+nn}{pd}
        \PY{k+kn}{import} \PY{n+nn}{numpy} \PY{k}{as} \PY{n+nn}{np}
        \PY{k+kn}{import} \PY{n+nn}{matplotlib}
        \PY{k+kn}{import} \PY{n+nn}{matplotlib}\PY{n+nn}{.}\PY{n+nn}{pyplot} \PY{k}{as} \PY{n+nn}{plt}
        \PY{k+kn}{import} \PY{n+nn}{seaborn} \PY{k}{as} \PY{n+nn}{sns}
        \PY{o}{\PYZpc{}}\PY{k}{matplotlib} inline
\end{Verbatim}


    \begin{Verbatim}[commandchars=\\\{\}]
{\color{incolor}In [{\color{incolor}2}]:} \PY{c+c1}{\PYZsh{} Importing items file}
        \PY{n}{items} \PY{o}{=} \PY{n}{pd}\PY{o}{.}\PY{n}{read\PYZus{}csv}\PY{p}{(}\PY{l+s+s1}{\PYZsq{}}\PY{l+s+s1}{items.csv}\PY{l+s+s1}{\PYZsq{}}\PY{p}{)}
        
        \PY{c+c1}{\PYZsh{} Importing users file}
        \PY{n}{users} \PY{o}{=} \PY{n}{pd}\PY{o}{.}\PY{n}{read\PYZus{}csv}\PY{p}{(}\PY{l+s+s1}{\PYZsq{}}\PY{l+s+s1}{users.csv}\PY{l+s+s1}{\PYZsq{}}\PY{p}{)}
        
        \PY{c+c1}{\PYZsh{} Importing ratings file}
        \PY{n}{ratings} \PY{o}{=} \PY{n}{pd}\PY{o}{.}\PY{n}{read\PYZus{}csv}\PY{p}{(}\PY{l+s+s1}{\PYZsq{}}\PY{l+s+s1}{ratings.csv}\PY{l+s+s1}{\PYZsq{}}\PY{p}{)}
\end{Verbatim}


    \begin{Verbatim}[commandchars=\\\{\}]
{\color{incolor}In [{\color{incolor}3}]:} \PY{c+c1}{\PYZsh{} Printing the shape of the items dataframe}
        \PY{n+nb}{print}\PY{p}{(}\PY{n}{items}\PY{o}{.}\PY{n}{shape}\PY{p}{)}
        \PY{c+c1}{\PYZsh{} It shows that there are 1682 rows of different movies, described by 23 columns}
        
        \PY{c+c1}{\PYZsh{} Prints the first 5 rows of the dataframe}
        \PY{n}{items}\PY{o}{.}\PY{n}{head}\PY{p}{(}\PY{p}{)}
\end{Verbatim}


    \begin{Verbatim}[commandchars=\\\{\}]
(1682, 24)

    \end{Verbatim}

\begin{Verbatim}[commandchars=\\\{\}]
{\color{outcolor}Out[{\color{outcolor}3}]:}    movie\_id              title release\_date  video\_release\_date  \textbackslash{}
        0         1   Toy Story (1995)  01-Jan-1995                 NaN   
        1         2   GoldenEye (1995)  01-Jan-1995                 NaN   
        2         3  Four Rooms (1995)  01-Jan-1995                 NaN   
        3         4  Get Shorty (1995)  01-Jan-1995                 NaN   
        4         5     Copycat (1995)  01-Jan-1995                 NaN   
        
                                                    imdb\_url  unknown  Action  \textbackslash{}
        0  http://us.imdb.com/M/title-exact?Toy\%20Story\%2{\ldots}        0       0   
        1  http://us.imdb.com/M/title-exact?GoldenEye\%20({\ldots}        0       1   
        2  http://us.imdb.com/M/title-exact?Four\%20Rooms\%{\ldots}        0       0   
        3  http://us.imdb.com/M/title-exact?Get\%20Shorty\%{\ldots}        0       1   
        4  http://us.imdb.com/M/title-exact?Copycat\%20(1995)        0       0   
        
           Adventure  Animation  Children's   {\ldots}     Fantasy  Film-Noir  Horror  \textbackslash{}
        0          0          1           1   {\ldots}           0          0       0   
        1          1          0           0   {\ldots}           0          0       0   
        2          0          0           0   {\ldots}           0          0       0   
        3          0          0           0   {\ldots}           0          0       0   
        4          0          0           0   {\ldots}           0          0       0   
        
           Musical  Mystery  Romance  Sci-Fi  Thriller  War  Western  
        0        0        0        0       0         0    0        0  
        1        0        0        0       0         1    0        0  
        2        0        0        0       0         1    0        0  
        3        0        0        0       0         0    0        0  
        4        0        0        0       0         1    0        0  
        
        [5 rows x 24 columns]
\end{Verbatim}
            
    \begin{Verbatim}[commandchars=\\\{\}]
{\color{incolor}In [{\color{incolor}4}]:} \PY{c+c1}{\PYZsh{} Printing the shape of the users dataframe}
        \PY{n+nb}{print}\PY{p}{(}\PY{n}{users}\PY{o}{.}\PY{n}{shape}\PY{p}{)}
        \PY{c+c1}{\PYZsh{} It shows that there are 943 rows of unique users, described by 4 columns}
        
        \PY{c+c1}{\PYZsh{} Prints the first 5 rows of the users dataframe}
        \PY{n}{users}\PY{o}{.}\PY{n}{head}\PY{p}{(}\PY{p}{)}
\end{Verbatim}


    \begin{Verbatim}[commandchars=\\\{\}]
(943, 5)

    \end{Verbatim}

\begin{Verbatim}[commandchars=\\\{\}]
{\color{outcolor}Out[{\color{outcolor}4}]:}    user\_id  age sex  occupation zip\_code
        0        1   24   M  technician    85711
        1        2   53   F       other    94043
        2        3   23   M      writer    32067
        3        4   24   M  technician    43537
        4        5   33   F       other    15213
\end{Verbatim}
            
    \begin{Verbatim}[commandchars=\\\{\}]
{\color{incolor}In [{\color{incolor}5}]:} \PY{c+c1}{\PYZsh{} Printing the shape of the ratings dataframe}
        \PY{n+nb}{print}\PY{p}{(}\PY{n}{ratings}\PY{o}{.}\PY{n}{shape}\PY{p}{)}
        \PY{c+c1}{\PYZsh{} It shows that there are 100000 rows of unique ratings, described by 3 columns. The ratings are all made by the 943 users.}
        
        \PY{c+c1}{\PYZsh{} Prints the first 5 rows of the ratings dataframe}
        \PY{n}{ratings}\PY{o}{.}\PY{n}{head}\PY{p}{(}\PY{p}{)}
        \PY{c+c1}{\PYZsh{} Ratings are from 1 to 5. Timestamps are the time when the user left the rating. They are unix timestamps, which are expressed}
        \PY{c+c1}{\PYZsh{} in seconds after 1970\PYZhy{}01\PYZhy{}01 00:00:00 UTC}
\end{Verbatim}


    \begin{Verbatim}[commandchars=\\\{\}]
(100000, 4)

    \end{Verbatim}

\begin{Verbatim}[commandchars=\\\{\}]
{\color{outcolor}Out[{\color{outcolor}5}]:}    user\_id  movie\_id  rating  unix\_timestamp
        0      196       242       3       881250949
        1      186       302       3       891717742
        2       22       377       1       878887116
        3      244        51       2       880606923
        4      166       346       1       886397596
\end{Verbatim}
            
    \begin{Verbatim}[commandchars=\\\{\}]
{\color{incolor}In [{\color{incolor}6}]:} \PY{c+c1}{\PYZsh{} items.describe() is not useful in this case, doesn\PYZsq{}t provide any useful information}
        \PY{c+c1}{\PYZsh{} items.describe()}
\end{Verbatim}


    \begin{Verbatim}[commandchars=\\\{\}]
{\color{incolor}In [{\color{incolor}7}]:} \PY{c+c1}{\PYZsh{} Describes the numerical values in the users dataframe, since there is only 1 numerical value, namely age, it is the}
        \PY{c+c1}{\PYZsh{} only one that is displayed}
        \PY{n}{users}\PY{o}{.}\PY{n}{describe}\PY{p}{(}\PY{p}{)}
\end{Verbatim}


\begin{Verbatim}[commandchars=\\\{\}]
{\color{outcolor}Out[{\color{outcolor}7}]:}           user\_id         age
        count  943.000000  943.000000
        mean   472.000000   34.051962
        std    272.364951   12.192740
        min      1.000000    7.000000
        25\%    236.500000   25.000000
        50\%    472.000000   31.000000
        75\%    707.500000   43.000000
        max    943.000000   73.000000
\end{Verbatim}
            
    \begin{Verbatim}[commandchars=\\\{\}]
{\color{incolor}In [{\color{incolor}8}]:} \PY{c+c1}{\PYZsh{} We can see that the lowest rating is 1 and the highest is 5, with an average of \PYZti{}3.5}
        \PY{n}{ratings}\PY{p}{[}\PY{l+s+s1}{\PYZsq{}}\PY{l+s+s1}{rating}\PY{l+s+s1}{\PYZsq{}}\PY{p}{]}\PY{o}{.}\PY{n}{describe}\PY{p}{(}\PY{p}{)}
\end{Verbatim}


\begin{Verbatim}[commandchars=\\\{\}]
{\color{outcolor}Out[{\color{outcolor}8}]:} count    100000.000000
        mean          3.529860
        std           1.125674
        min           1.000000
        25\%           3.000000
        50\%           4.000000
        75\%           4.000000
        max           5.000000
        Name: rating, dtype: float64
\end{Verbatim}
            
    \begin{Verbatim}[commandchars=\\\{\}]
{\color{incolor}In [{\color{incolor}9}]:} \PY{c+c1}{\PYZsh{} Merge items (movies) dataframe with ratings dataframe on common column movie\PYZus{}id}
        \PY{n}{movie\PYZus{}ratings} \PY{o}{=} \PY{n}{pd}\PY{o}{.}\PY{n}{merge}\PY{p}{(}\PY{n}{items}\PY{p}{,} \PY{n}{ratings}\PY{p}{,} \PY{n}{on}\PY{o}{=}\PY{l+s+s1}{\PYZsq{}}\PY{l+s+s1}{movie\PYZus{}id}\PY{l+s+s1}{\PYZsq{}}\PY{p}{)}
        \PY{n}{movie\PYZus{}ratings}\PY{o}{.}\PY{n}{head}\PY{p}{(}\PY{p}{)}
\end{Verbatim}


\begin{Verbatim}[commandchars=\\\{\}]
{\color{outcolor}Out[{\color{outcolor}9}]:}    movie\_id             title release\_date  video\_release\_date  \textbackslash{}
        0         1  Toy Story (1995)  01-Jan-1995                 NaN   
        1         1  Toy Story (1995)  01-Jan-1995                 NaN   
        2         1  Toy Story (1995)  01-Jan-1995                 NaN   
        3         1  Toy Story (1995)  01-Jan-1995                 NaN   
        4         1  Toy Story (1995)  01-Jan-1995                 NaN   
        
                                                    imdb\_url  unknown  Action  \textbackslash{}
        0  http://us.imdb.com/M/title-exact?Toy\%20Story\%2{\ldots}        0       0   
        1  http://us.imdb.com/M/title-exact?Toy\%20Story\%2{\ldots}        0       0   
        2  http://us.imdb.com/M/title-exact?Toy\%20Story\%2{\ldots}        0       0   
        3  http://us.imdb.com/M/title-exact?Toy\%20Story\%2{\ldots}        0       0   
        4  http://us.imdb.com/M/title-exact?Toy\%20Story\%2{\ldots}        0       0   
        
           Adventure  Animation  Children's       {\ldots}        Musical  Mystery  \textbackslash{}
        0          0          1           1       {\ldots}              0        0   
        1          0          1           1       {\ldots}              0        0   
        2          0          1           1       {\ldots}              0        0   
        3          0          1           1       {\ldots}              0        0   
        4          0          1           1       {\ldots}              0        0   
        
           Romance  Sci-Fi  Thriller  War  Western  user\_id  rating  unix\_timestamp  
        0        0       0         0    0        0      308       4       887736532  
        1        0       0         0    0        0      287       5       875334088  
        2        0       0         0    0        0      148       4       877019411  
        3        0       0         0    0        0      280       4       891700426  
        4        0       0         0    0        0       66       3       883601324  
        
        [5 rows x 27 columns]
\end{Verbatim}
            
    \begin{Verbatim}[commandchars=\\\{\}]
{\color{incolor}In [{\color{incolor}10}]:} \PY{c+c1}{\PYZsh{} Create number of ratings column per movie in ratings dataset}
         \PY{n}{ratings\PYZus{}average} \PY{o}{=} \PY{n}{pd}\PY{o}{.}\PY{n}{DataFrame}\PY{p}{(}\PY{n}{movie\PYZus{}ratings}\PY{o}{.}\PY{n}{groupby}\PY{p}{(}\PY{l+s+s1}{\PYZsq{}}\PY{l+s+s1}{title}\PY{l+s+s1}{\PYZsq{}}\PY{p}{)}\PY{p}{[}\PY{l+s+s1}{\PYZsq{}}\PY{l+s+s1}{rating}\PY{l+s+s1}{\PYZsq{}}\PY{p}{]}\PY{o}{.}\PY{n}{mean}\PY{p}{(}\PY{p}{)}\PY{p}{)}
         \PY{n}{ratings\PYZus{}average}\PY{o}{.}\PY{n}{head}\PY{p}{(}\PY{p}{)}
\end{Verbatim}


\begin{Verbatim}[commandchars=\\\{\}]
{\color{outcolor}Out[{\color{outcolor}10}]:}                              rating
         title                              
         'Til There Was You (1997)  2.333333
         1-900 (1994)               2.600000
         101 Dalmatians (1996)      2.908257
         12 Angry Men (1957)        4.344000
         187 (1997)                 3.024390
\end{Verbatim}
            
    \begin{Verbatim}[commandchars=\\\{\}]
{\color{incolor}In [{\color{incolor}11}]:} \PY{c+c1}{\PYZsh{} Plot average rating in a histogram}
         \PY{n}{ratings\PYZus{}average}\PY{p}{[}\PY{l+s+s1}{\PYZsq{}}\PY{l+s+s1}{rating}\PY{l+s+s1}{\PYZsq{}}\PY{p}{]}\PY{o}{.}\PY{n}{hist}\PY{p}{(}\PY{n}{bins}\PY{o}{=}\PY{l+m+mi}{50}\PY{p}{)}  
\end{Verbatim}


\begin{Verbatim}[commandchars=\\\{\}]
{\color{outcolor}Out[{\color{outcolor}11}]:} <matplotlib.axes.\_subplots.AxesSubplot at 0x7f5d77709a20>
\end{Verbatim}
            
    \begin{center}
    \adjustimage{max size={0.9\linewidth}{0.9\paperheight}}{output_10_1.png}
    \end{center}
    { \hspace*{\fill} \\}
    
    \begin{Verbatim}[commandchars=\\\{\}]
{\color{incolor}In [{\color{incolor}12}]:} \PY{c+c1}{\PYZsh{} Create a number of ratings column to see how many ratings each movie has}
         \PY{n}{ratings\PYZus{}average}\PY{p}{[}\PY{l+s+s1}{\PYZsq{}}\PY{l+s+s1}{number\PYZus{}of\PYZus{}ratings}\PY{l+s+s1}{\PYZsq{}}\PY{p}{]} \PY{o}{=} \PY{n}{movie\PYZus{}ratings}\PY{o}{.}\PY{n}{groupby}\PY{p}{(}\PY{l+s+s1}{\PYZsq{}}\PY{l+s+s1}{title}\PY{l+s+s1}{\PYZsq{}}\PY{p}{)}\PY{p}{[}\PY{l+s+s1}{\PYZsq{}}\PY{l+s+s1}{rating}\PY{l+s+s1}{\PYZsq{}}\PY{p}{]}\PY{o}{.}\PY{n}{count}\PY{p}{(}\PY{p}{)}
\end{Verbatim}


    \begin{Verbatim}[commandchars=\\\{\}]
{\color{incolor}In [{\color{incolor}13}]:} \PY{c+c1}{\PYZsh{} Dataframe after new column}
         \PY{n}{ratings\PYZus{}average}\PY{o}{.}\PY{n}{head}\PY{p}{(}\PY{p}{)}
\end{Verbatim}


\begin{Verbatim}[commandchars=\\\{\}]
{\color{outcolor}Out[{\color{outcolor}13}]:}                              rating  number\_of\_ratings
         title                                                 
         'Til There Was You (1997)  2.333333                  9
         1-900 (1994)               2.600000                  5
         101 Dalmatians (1996)      2.908257                109
         12 Angry Men (1957)        4.344000                125
         187 (1997)                 3.024390                 41
\end{Verbatim}
            
    \begin{Verbatim}[commandchars=\\\{\}]
{\color{incolor}In [{\color{incolor}14}]:} \PY{c+c1}{\PYZsh{} Display movies with most ratings}
         \PY{n}{ratings\PYZus{}average}\PY{o}{.}\PY{n}{sort\PYZus{}values}\PY{p}{(}\PY{l+s+s1}{\PYZsq{}}\PY{l+s+s1}{number\PYZus{}of\PYZus{}ratings}\PY{l+s+s1}{\PYZsq{}}\PY{p}{,} \PY{n}{ascending}\PY{o}{=}\PY{k+kc}{False}\PY{p}{)}\PY{o}{.}\PY{n}{head}\PY{p}{(}\PY{l+m+mi}{5}\PY{p}{)}
\end{Verbatim}


\begin{Verbatim}[commandchars=\\\{\}]
{\color{outcolor}Out[{\color{outcolor}14}]:}                              rating  number\_of\_ratings
         title                                                 
         Star Wars (1977)           4.358491                583
         Contact (1997)             3.803536                509
         Fargo (1996)               4.155512                508
         Return of the Jedi (1983)  4.007890                507
         Liar Liar (1997)           3.156701                485
\end{Verbatim}
            
    \begin{Verbatim}[commandchars=\\\{\}]
{\color{incolor}In [{\color{incolor}15}]:} \PY{c+c1}{\PYZsh{} Plot a histogram of the number of ratings to roughly see the distribution of the count of ratings }
         \PY{n}{ratings\PYZus{}average}\PY{p}{[}\PY{l+s+s1}{\PYZsq{}}\PY{l+s+s1}{number\PYZus{}of\PYZus{}ratings}\PY{l+s+s1}{\PYZsq{}}\PY{p}{]}\PY{o}{.}\PY{n}{hist}\PY{p}{(}\PY{n}{bins}\PY{o}{=}\PY{l+m+mi}{50}\PY{p}{)}
         \PY{c+c1}{\PYZsh{} It can be seen that most movies have few ratings}
\end{Verbatim}


\begin{Verbatim}[commandchars=\\\{\}]
{\color{outcolor}Out[{\color{outcolor}15}]:} <matplotlib.axes.\_subplots.AxesSubplot at 0x7f5d77652438>
\end{Verbatim}
            
    \begin{center}
    \adjustimage{max size={0.9\linewidth}{0.9\paperheight}}{output_14_1.png}
    \end{center}
    { \hspace*{\fill} \\}
    
    \begin{Verbatim}[commandchars=\\\{\}]
{\color{incolor}In [{\color{incolor}16}]:} \PY{c+c1}{\PYZsh{} Is there a correlation between the number of ratings and rating itself?}
         \PY{n}{ratings\PYZus{}average}\PY{o}{.}\PY{n}{corr}\PY{p}{(}\PY{p}{)}
         \PY{c+c1}{\PYZsh{} There seems to be quite a big correlation between the 2 variables \PYZhy{} 43\PYZpc{}}
\end{Verbatim}


\begin{Verbatim}[commandchars=\\\{\}]
{\color{outcolor}Out[{\color{outcolor}16}]:}                      rating  number\_of\_ratings
         rating             1.000000           0.430998
         number\_of\_ratings  0.430998           1.000000
\end{Verbatim}
            
    \begin{Verbatim}[commandchars=\\\{\}]
{\color{incolor}In [{\color{incolor}17}]:} \PY{c+c1}{\PYZsh{} See the distribution of number of ratings and average ratings}
         \PY{n}{sns}\PY{o}{.}\PY{n}{jointplot}\PY{p}{(}\PY{n}{x}\PY{o}{=}\PY{l+s+s1}{\PYZsq{}}\PY{l+s+s1}{rating}\PY{l+s+s1}{\PYZsq{}}\PY{p}{,} \PY{n}{y}\PY{o}{=}\PY{l+s+s1}{\PYZsq{}}\PY{l+s+s1}{number\PYZus{}of\PYZus{}ratings}\PY{l+s+s1}{\PYZsq{}}\PY{p}{,} \PY{n}{data}\PY{o}{=}\PY{n}{ratings\PYZus{}average}\PY{p}{,} \PY{n}{kind}\PY{o}{=}\PY{l+s+s2}{\PYZdq{}}\PY{l+s+s2}{reg}\PY{l+s+s2}{\PYZdq{}}\PY{p}{,} \PY{n}{color}\PY{o}{=}\PY{l+s+s2}{\PYZdq{}}\PY{l+s+s2}{g}\PY{l+s+s2}{\PYZdq{}}\PY{p}{)}  
\end{Verbatim}


    \begin{Verbatim}[commandchars=\\\{\}]
/home/vanko/anaconda3/lib/python3.6/site-packages/matplotlib/axes/\_axes.py:6462: UserWarning: The 'normed' kwarg is deprecated, and has been replaced by the 'density' kwarg.
  warnings.warn("The 'normed' kwarg is deprecated, and has been "
/home/vanko/anaconda3/lib/python3.6/site-packages/matplotlib/axes/\_axes.py:6462: UserWarning: The 'normed' kwarg is deprecated, and has been replaced by the 'density' kwarg.
  warnings.warn("The 'normed' kwarg is deprecated, and has been "

    \end{Verbatim}

\begin{Verbatim}[commandchars=\\\{\}]
{\color{outcolor}Out[{\color{outcolor}17}]:} <seaborn.axisgrid.JointGrid at 0x7f5d77525b00>
\end{Verbatim}
            
    \begin{center}
    \adjustimage{max size={0.9\linewidth}{0.9\paperheight}}{output_16_2.png}
    \end{center}
    { \hspace*{\fill} \\}
    
    \section{Using the correlation between the ratings of movies as a
similarity
metric}\label{using-the-correlation-between-the-ratings-of-movies-as-a-similarity-metric}

    \begin{Verbatim}[commandchars=\\\{\}]
{\color{incolor}In [{\color{incolor}18}]:} \PY{c+c1}{\PYZsh{} Create a pivot table to display every rating (in table) of each user (vertical) for each movie (horizontal)}
         \PY{n}{user\PYZus{}movie\PYZus{}rating} \PY{o}{=} \PY{n}{movie\PYZus{}ratings}\PY{o}{.}\PY{n}{pivot\PYZus{}table}\PY{p}{(}\PY{n}{index}\PY{o}{=}\PY{l+s+s1}{\PYZsq{}}\PY{l+s+s1}{user\PYZus{}id}\PY{l+s+s1}{\PYZsq{}}\PY{p}{,} \PY{n}{columns}\PY{o}{=}\PY{l+s+s1}{\PYZsq{}}\PY{l+s+s1}{title}\PY{l+s+s1}{\PYZsq{}}\PY{p}{,} \PY{n}{values}\PY{o}{=}\PY{l+s+s1}{\PYZsq{}}\PY{l+s+s1}{rating}\PY{l+s+s1}{\PYZsq{}}\PY{p}{)}
         \PY{n}{user\PYZus{}movie\PYZus{}rating}\PY{o}{.}\PY{n}{head}\PY{p}{(}\PY{p}{)}
\end{Verbatim}


\begin{Verbatim}[commandchars=\\\{\}]
{\color{outcolor}Out[{\color{outcolor}18}]:} title    'Til There Was You (1997)  1-900 (1994)  101 Dalmatians (1996)  \textbackslash{}
         user\_id                                                                   
         1                              NaN           NaN                    2.0   
         2                              NaN           NaN                    NaN   
         3                              NaN           NaN                    NaN   
         4                              NaN           NaN                    NaN   
         5                              NaN           NaN                    2.0   
         
         title    12 Angry Men (1957)  187 (1997)  2 Days in the Valley (1996)  \textbackslash{}
         user\_id                                                                 
         1                        5.0         NaN                          NaN   
         2                        NaN         NaN                          NaN   
         3                        NaN         2.0                          NaN   
         4                        NaN         NaN                          NaN   
         5                        NaN         NaN                          NaN   
         
         title    20,000 Leagues Under the Sea (1954)  2001: A Space Odyssey (1968)  \textbackslash{}
         user\_id                                                                      
         1                                        3.0                           4.0   
         2                                        NaN                           NaN   
         3                                        NaN                           NaN   
         4                                        NaN                           NaN   
         5                                        NaN                           4.0   
         
         title    3 Ninjas: High Noon At Mega Mountain (1998)  39 Steps, The (1935)  \textbackslash{}
         user\_id                                                                      
         1                                                NaN                   NaN   
         2                                                1.0                   NaN   
         3                                                NaN                   NaN   
         4                                                NaN                   NaN   
         5                                                NaN                   NaN   
         
         title                   {\ldots}                  Yankee Zulu (1994)  \textbackslash{}
         user\_id                 {\ldots}                                       
         1                       {\ldots}                                 NaN   
         2                       {\ldots}                                 NaN   
         3                       {\ldots}                                 NaN   
         4                       {\ldots}                                 NaN   
         5                       {\ldots}                                 NaN   
         
         title    Year of the Horse (1997)  You So Crazy (1994)  \textbackslash{}
         user\_id                                                  
         1                             NaN                  NaN   
         2                             NaN                  NaN   
         3                             NaN                  NaN   
         4                             NaN                  NaN   
         5                             NaN                  NaN   
         
         title    Young Frankenstein (1974)  Young Guns (1988)  Young Guns II (1990)  \textbackslash{}
         user\_id                                                                       
         1                              5.0                3.0                   NaN   
         2                              NaN                NaN                   NaN   
         3                              NaN                NaN                   NaN   
         4                              NaN                NaN                   NaN   
         5                              4.0                NaN                   NaN   
         
         title    Young Poisoner's Handbook, The (1995)  Zeus and Roxanne (1997)  \textbackslash{}
         user\_id                                                                   
         1                                          NaN                      NaN   
         2                                          NaN                      NaN   
         3                                          NaN                      NaN   
         4                                          NaN                      NaN   
         5                                          NaN                      NaN   
         
         title    unknown  Á köldum klaka (Cold Fever) (1994)  
         user\_id                                               
         1            4.0                                 NaN  
         2            NaN                                 NaN  
         3            NaN                                 NaN  
         4            NaN                                 NaN  
         5            4.0                                 NaN  
         
         [5 rows x 1664 columns]
\end{Verbatim}
            
    \begin{Verbatim}[commandchars=\\\{\}]
{\color{incolor}In [{\color{incolor}19}]:} \PY{c+c1}{\PYZsh{} Select Toy Story movie as an example (displays every rating by each user for Toy Story)}
         \PY{n}{toy\PYZus{}story\PYZus{}ratings} \PY{o}{=} \PY{n}{user\PYZus{}movie\PYZus{}rating}\PY{p}{[}\PY{l+s+s1}{\PYZsq{}}\PY{l+s+s1}{Toy Story (1995)}\PY{l+s+s1}{\PYZsq{}}\PY{p}{]}
         \PY{n}{toy\PYZus{}story\PYZus{}ratings}\PY{o}{.}\PY{n}{head}\PY{p}{(}\PY{p}{)}
\end{Verbatim}


\begin{Verbatim}[commandchars=\\\{\}]
{\color{outcolor}Out[{\color{outcolor}19}]:} user\_id
         1    5.0
         2    4.0
         3    NaN
         4    NaN
         5    4.0
         Name: Toy Story (1995), dtype: float64
\end{Verbatim}
            
    \begin{Verbatim}[commandchars=\\\{\}]
{\color{incolor}In [{\color{incolor}20}]:} \PY{c+c1}{\PYZsh{} Find the correlation between Toy Story and every other movie in the dataset}
         \PY{n}{movies\PYZus{}like\PYZus{}toy\PYZus{}story} \PY{o}{=} \PY{n}{user\PYZus{}movie\PYZus{}rating}\PY{o}{.}\PY{n}{corrwith}\PY{p}{(}\PY{n}{toy\PYZus{}story\PYZus{}ratings}\PY{p}{)}
         \PY{c+c1}{\PYZsh{} Create a dataframe out of the correlations}
         \PY{n}{corr\PYZus{}toy\PYZus{}story} \PY{o}{=} \PY{n}{pd}\PY{o}{.}\PY{n}{DataFrame}\PY{p}{(}\PY{n}{movies\PYZus{}like\PYZus{}toy\PYZus{}story}\PY{p}{,} \PY{n}{columns}\PY{o}{=}\PY{p}{[}\PY{l+s+s1}{\PYZsq{}}\PY{l+s+s1}{correlation}\PY{l+s+s1}{\PYZsq{}}\PY{p}{]}\PY{p}{)}
         \PY{c+c1}{\PYZsh{} Join ratings\PYZus{}average dataframe to see how many ratings each correlated movie has}
         \PY{n}{corr\PYZus{}toy\PYZus{}story} \PY{o}{=} \PY{n}{corr\PYZus{}toy\PYZus{}story}\PY{o}{.}\PY{n}{join}\PY{p}{(}\PY{n}{ratings\PYZus{}average}\PY{p}{[}\PY{l+s+s1}{\PYZsq{}}\PY{l+s+s1}{number\PYZus{}of\PYZus{}ratings}\PY{l+s+s1}{\PYZsq{}}\PY{p}{]}\PY{p}{)}  
         \PY{c+c1}{\PYZsh{} Drop all rows with NaN values}
         \PY{n}{corr\PYZus{}toy\PYZus{}story}\PY{o}{.}\PY{n}{dropna}\PY{p}{(}\PY{n}{inplace}\PY{o}{=}\PY{k+kc}{True}\PY{p}{)}  
         \PY{n}{corr\PYZus{}toy\PYZus{}story}\PY{o}{.}\PY{n}{head}\PY{p}{(}\PY{p}{)}  
\end{Verbatim}


    \begin{Verbatim}[commandchars=\\\{\}]
/home/vanko/anaconda3/lib/python3.6/site-packages/numpy/lib/function\_base.py:3175: RuntimeWarning: Degrees of freedom <= 0 for slice
  c = cov(x, y, rowvar)
/home/vanko/anaconda3/lib/python3.6/site-packages/numpy/lib/function\_base.py:3109: RuntimeWarning: divide by zero encountered in double\_scalars
  c *= 1. / np.float64(fact)

    \end{Verbatim}

\begin{Verbatim}[commandchars=\\\{\}]
{\color{outcolor}Out[{\color{outcolor}20}]:}                              correlation  number\_of\_ratings
         title                                                      
         'Til There Was You (1997)       0.534522                  9
         101 Dalmatians (1996)           0.232118                109
         12 Angry Men (1957)             0.334943                125
         187 (1997)                      0.651857                 41
         2 Days in the Valley (1996)     0.162728                 93
\end{Verbatim}
            
    \begin{Verbatim}[commandchars=\\\{\}]
{\color{incolor}In [{\color{incolor}21}]:} \PY{c+c1}{\PYZsh{} Display 10 most correlated movies}
         \PY{n}{corr\PYZus{}toy\PYZus{}story}\PY{o}{.}\PY{n}{sort\PYZus{}values}\PY{p}{(}\PY{l+s+s1}{\PYZsq{}}\PY{l+s+s1}{correlation}\PY{l+s+s1}{\PYZsq{}}\PY{p}{,} \PY{n}{ascending}\PY{o}{=}\PY{k+kc}{False}\PY{p}{)}\PY{o}{.}\PY{n}{head}\PY{p}{(}\PY{l+m+mi}{10}\PY{p}{)}
         \PY{c+c1}{\PYZsh{} It can be seen that all movies that have 100\PYZpc{} correlation have a very few number of ratings, from which we cannot}
         \PY{c+c1}{\PYZsh{} deduce similarity}
\end{Verbatim}


\begin{Verbatim}[commandchars=\\\{\}]
{\color{outcolor}Out[{\color{outcolor}21}]:}                                                     correlation  \textbackslash{}
         title                                                             
         Old Lady Who Walked in the Sea, The (Vieille qu{\ldots}          1.0   
         Reckless (1995)                                             1.0   
         Ladybird Ladybird (1994)                                    1.0   
         Infinity (1996)                                             1.0   
         Albino Alligator (1996)                                     1.0   
         Toy Story (1995)                                            1.0   
         Guantanamera (1994)                                         1.0   
         Late Bloomers (1996)                                        1.0   
         Across the Sea of Time (1995)                               1.0   
         Substance of Fire, The (1996)                               1.0   
         
                                                             number\_of\_ratings  
         title                                                                  
         Old Lady Who Walked in the Sea, The (Vieille qu{\ldots}                  5  
         Reckless (1995)                                                     8  
         Ladybird Ladybird (1994)                                            4  
         Infinity (1996)                                                     6  
         Albino Alligator (1996)                                             6  
         Toy Story (1995)                                                  452  
         Guantanamera (1994)                                                 4  
         Late Bloomers (1996)                                                5  
         Across the Sea of Time (1995)                                       4  
         Substance of Fire, The (1996)                                       4  
\end{Verbatim}
            
    \begin{Verbatim}[commandchars=\\\{\}]
{\color{incolor}In [{\color{incolor}22}]:} \PY{c+c1}{\PYZsh{} Therefore we need to put a condition on a minimum number of ratings}
         \PY{n}{corr\PYZus{}toy\PYZus{}story}\PY{p}{[}\PY{n}{corr\PYZus{}toy\PYZus{}story}\PY{p}{[}\PY{l+s+s1}{\PYZsq{}}\PY{l+s+s1}{number\PYZus{}of\PYZus{}ratings}\PY{l+s+s1}{\PYZsq{}}\PY{p}{]}\PY{o}{\PYZgt{}}\PY{l+m+mi}{50}\PY{p}{]}\PY{o}{.}\PY{n}{sort\PYZus{}values}\PY{p}{(}\PY{l+s+s1}{\PYZsq{}}\PY{l+s+s1}{correlation}\PY{l+s+s1}{\PYZsq{}}\PY{p}{,} \PY{n}{ascending}\PY{o}{=}\PY{k+kc}{False}\PY{p}{)}\PY{o}{.}\PY{n}{head}\PY{p}{(}\PY{p}{)}  
\end{Verbatim}


\begin{Verbatim}[commandchars=\\\{\}]
{\color{outcolor}Out[{\color{outcolor}22}]:}                               correlation  number\_of\_ratings
         title                                                       
         Toy Story (1995)                 1.000000                452
         Raise the Red Lantern (1991)     0.641535                 58
         Flubber (1997)                   0.558389                 53
         Jackal, The (1997)               0.557876                 87
         Craft, The (1996)                0.549100                104
\end{Verbatim}
            

    % Add a bibliography block to the postdoc
    
    
    
    \end{document}
